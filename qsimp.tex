% coding:utf-8

%FOSAET, a LaTeX-Code for a electrical summary of basic electronics
%Copyright (C) 2013, Daniel Winz, Ervin Mazlagic

%This program is free software; you can redistribute it and/or
%modify it under the terms of the GNU General Public License
%as published by the Free Software Foundation; either version 2
%of the License, or (at your option) any later version.

%This program is distributed in the hope that it will be useful,
%but WITHOUT ANY WARRANTY; without even the implied warranty of
%MERCHANTABILITY or FITNESS FOR A PARTICULAR PURPOSE.  See the
%GNU General Public License for more details.
%----------------------------------------

\section{Quellenvereinfachung}
Interessiert die Leistung an der Quelle nicht oder ist die Spannung an 
Stromquellen oder der Strom durch Spannungsquellen nicht wichtig, können 
Quellen wie folgt vereinfacht werden. 
\begin{center}
\begin{minipage}[c]{0.3\textwidth}
\begin{circuitikz}[scale=1]\draw
  (0,5) to[V=$U_q$, o-] (0,3)
  (0,3) to[I=$I_q$, -o] (0,1)
  ;
\end{circuitikz}
\end{minipage}
\begin{minipage}[c]{0.1\textwidth}
\Huge$\Rightarrow$
\end{minipage}
\begin{minipage}[c]{0.3\textwidth}
\begin{circuitikz}[scale=1]\draw
  (2,4) to[I=$I_q$, o-o] (2,2)
  ;
\end{circuitikz}
\end{minipage}
\end{center}
%
\begin{center}
\begin{minipage}[c]{0.3\textwidth}
\begin{circuitikz}[scale=1]\draw
  (1,4) to[short, o-*] (1,3)
  (0,3) to[short, ] (2,3)
  (0,1) to[short, ] (2,1)
  (1,1) to[short, *-o] (1,0)
  (0,3) to[V=$U_q$, ] (0,1)
  (2,3) to[I=$I_q$, ] (2,1)
  ;
\end{circuitikz}
\end{minipage}
\begin{minipage}[c]{0.1\textwidth}
\Huge$\Rightarrow$
\end{minipage}
\begin{minipage}[c]{0.3\textwidth}
\begin{circuitikz}[scale=1]\draw
  (4,3) to[V=$U_q$, o-o] (4,1)
  ;
\end{circuitikz}
\end{minipage}
\end{center}
%
\begin{center}
\begin{minipage}[c]{0.3\textwidth}
\begin{circuitikz}[scale=1]\draw
  (0,5) to[R=$R$, o-] (0,3)
  (0,3) to[I=$I_q$, -o] (0,1)
  ;
\end{circuitikz}
\end{minipage}
\begin{minipage}[c]{0.1\textwidth}
\Huge$\Rightarrow$
\end{minipage}
\begin{minipage}[c]{0.3\textwidth}
\begin{circuitikz}[scale=1]\draw
  (2,4) to[I=$I_q$, o-o] (2,2)
  ;
\end{circuitikz}
\end{minipage}
\end{center}
%
\begin{center}
\begin{minipage}[c]{0.3\textwidth}
\begin{circuitikz}[scale=1]\draw
  (1,4) to[short, o-*] (1,3)
  (0,3) to[short, ] (2,3)
  (0,1) to[short, ] (2,1)
  (1,1) to[short, *-o] (1,0)
  (0,3) to[V=$U_q$, ] (0,1)
  (2,3) to[R=$R$, ] (2,1)
  ;
\end{circuitikz}
\end{minipage}
\begin{minipage}[c]{0.1\textwidth}
\Huge$\Rightarrow$
\end{minipage}
\begin{minipage}[c]{0.3\textwidth}
\begin{circuitikz}[scale=1]\draw
  (4,3) to[V=$U_q$, o-o] (4,1)
  ;
\end{circuitikz}
\end{minipage}
\end{center}
%
