% coding:utf-8

%FOSAET, a LaTeX-Code for a electrical summary of basic electronics
%Copyright (C) 2013, Daniel Winz, Ervin Mazlagic

%This program is free software; you can redistribute it and/or
%modify it under the terms of the GNU General Public License
%as published by the Free Software Foundation; either version 2
%of the License, or (at your option) any later version.

%This program is distributed in the hope that it will be useful,
%but WITHOUT ANY WARRANTY; without even the implied warranty of
%MERCHANTABILITY or FITNESS FOR A PARTICULAR PURPOSE.  See the
%GNU General Public License for more details.
%----------------------------------------

\section{Filter höherer Ordnung}
Filter höherer Ordnung können aus Filtern erster und zweiter Ordnung 
zusammengesetzt werden. Dabei können die einzelnen Filter parallel oder in 
Kaskadenform kombiniert werden. Die Kaskadenform wird dabei weitaus häufiger 
verwendet. 

\subsection{Allgemeines Vorgehen}
\begin{enumerate}
  \item Ordnung bestimmen 
        (Seite \pageref{filt:ord})
  \item Koeffizienten bestimmen 
        (Seite \pageref{filt:koeff})
  \item Teilfilter dimensionieren 
        (Seiten \pageref{filt:o1-tp} bis  \pageref{filt:o2-bp})
\end{enumerate}

\subsection{Bestimmen der Ordnung}
\label{filt:ord}
Butterworth Filter
\[ N \geq \frac{\log_{10}\left(\dfrac{\lambda}{\epsilon}\right)}
{\log_{10}\left(\dfrac{f_s}{f_g}\right)} 
= \frac{\log_{10}\left(\sqrt{\dfrac{10^{(0.1 \cdot A_{min_{dB}})} - 1}
{10^{(0.1 \cdot A_{max_{dB}})} - 1}}\right)}
{\log_{10}\left(\dfrac{f_s}{f_g}\right)} 
= \frac{\log_{10}\left(\sqrt{\dfrac{\frac{1}{{A_{min}}^2} - 1}
{\frac{1}{{A_{max}}^2} - 1}}\right)}
{\log_{10}\left(\dfrac{f_s}{f_g}\right)} \]
Tschebyscheff Filter
\[ N \geq \frac{\arccosh\left(\dfrac{\lambda}{\epsilon}\right)}
{\arccosh\left(\dfrac{f_s}{f_g}\right)} 
= \frac{\arccosh\left(\sqrt{\dfrac{10^{(0.1 \cdot A_{min_{dB}})} - 1}
{10^{(0.1 \cdot A_{max_{dB}})} - 1}}\right)}
{\arccosh\left(\dfrac{f_s}{f_g}\right)} 
= \frac{\arccosh\left(\sqrt{\dfrac{\frac{1}{{A_{min}}^2} - 1}
{\frac{1}{{A_{max}}^2} - 1}}\right)}
{\arccosh\left(\dfrac{f_s}{f_g}\right)} \]
\begin{tabular}{ll}
$f_g$: & Grenzfrequenz \\
$f_s$: & Beginn des Sperrbereiches
\end{tabular}

\newpage
\subsection{Bestimmen der Koeffizienten}
\label{filt:koeff}
\begin{table}[h!]
\[ \begin{array}{@{}lcccccccc}
\rowcolor{white} n & f_{0_1}& Q_1   & f_{0_2}   & Q_2   & f_{0_3}   & Q_3   & f_{0_4}   & Q_4   \\
\rowcolor{lgray} 2 & 1.000  & 0.707 &           &       &           &       &           &       \\
\rowcolor{white} 3 & 1.000  & 1.000 & 1.000     &       &           &       &           &       \\
\rowcolor{lgray} 4 & 1.000  & 0.541 & 1.000     & 1.306 &           &       &           &       \\
\rowcolor{white} 5 & 1.000  & 0.618 & 1.000     & 1.620 & 1.000     &       &           &       \\
\rowcolor{lgray} 6 & 1.000  & 0.518 & 1.000     & 0.707 & 1.000     & 1.932 &           &       \\
\rowcolor{white} 7 & 1.000  & 0.555 & 1.000     & 0.802 & 1.000     & 2.247 & 1.000     &       \\
\rowcolor{lgray} 8 & 1.000  & 0.510 & 1.000     & 0.601 & 1.000     & 0.900 & 1.000     & 2.563 \\
\end{array} \]
\caption{Filterkoeffizienten für Butterworth, $A_{max} = 3 dB$}
\label{tab:filt-coeff-butt}
\end{table}

\begin{table}[h!]
\[ \begin{array}{@{}lcccccccc}
\rowcolor{white} n & f_{0_1}& Q_1   & f_{0_2}   & Q_2   & f_{0_3}   & Q_3   & f_{0_4}   & Q_4   \\
\rowcolor{lgray} 2 & 1.274 & 0.577  &           &       &           &       &           &       \\
\rowcolor{white} 3 & 1.453 & 0.691  & 1.327     &       &           &       &           &       \\
\rowcolor{lgray} 4 & 1.419 & 0.522  & 1.591     & 0.806 &           &       &           &       \\
\rowcolor{white} 5 & 1.561 & 0.564  & 1.760     & 0.917 & 1.507     &       &           &       \\
\rowcolor{lgray} 6 & 1.606 & 0.510  & 1.691     & 0.611 & 1.907     & 1.023 &           &       \\
\rowcolor{white} 7 & 1.719 & 0.533  & 1.824     & 0.661 & 2.051     & 1.127 & 1.685     &       \\
\rowcolor{lgray} 8 & 1.784 & 0.506  & 1.838     & 0.560 & 1.958     & 0.711 & 2.196     & 1.226 \\
\end{array} \]
\caption{Filterkoeffizienten für Bessel, $A_{max} = 3 dB$}
\label{tab:filt-coeff-bess}
\end{table}

\begin{table}[h!]
\[ \begin{array}{@{}lcccccccc}
\rowcolor{white} n & f_{0_1}& Q_1   & f_{0_2}   & Q_2   & f_{0_3}   & Q_3   & f_{0_4}   & Q_4   \\
\rowcolor{lgray} 2 & 1.820  & 0.767 &           &       &           &       &           &       \\
\rowcolor{white} 3 & 1.300  & 1.341 & 0.969     &       &           &       &           &       \\
\rowcolor{lgray} 4 & 1.153  & 2.183 & 0.789     & 0.619 &           &       &           &       \\
\rowcolor{white} 5 & 1.093  & 3.282 & 0.797     & 0.915 & 0.539     &       &           &       \\
\rowcolor{lgray} 6 & 1.063  & 4.633 & 0.834     & 1.332 & 0.513     & 0.599 &           &       \\
\rowcolor{white} 7 & 1.045  & 6.233 & 0.868     & 1.847 & 0.575     & 0.846 & 0.377     &       \\
\rowcolor{lgray} 8 & 1.034  & 8.082 & 0.894     & 2.453 & 0.645     & 1.183 & 0.382     & 0.593 \\
\end{array} \]
\caption{Filterkoeffizienten für Tschebyscheff mit $0.1 dB$ Welligkeit}
\label{tab:filt-coeff-tsche01}
\end{table}

\begin{table}[h!]
\[ \begin{array}{@{}lcccccccc}
\rowcolor{white} n & f_{0_1}& Q_1   & f_{0_2}   & Q_2   & f_{0_3}   & Q_3   & f_{0_4}   & Q_4   \\
\rowcolor{lgray} 2 & 1.050  & 0.957 &           &       &           &       &           &       \\
\rowcolor{white} 3 & 0.997  & 2.018 & 0.494     &       &           &       &           &       \\
\rowcolor{lgray} 4 & 0.993  & 3.559 & 0.529     & 0.785 &           &       &           &       \\
\rowcolor{white} 5 & 0.994  & 5.559 & 0.655     & 1.399 & 0.289     &       &           &       \\
\rowcolor{lgray} 6 & 0.995  & 8.004 & 0.747     & 2.198 & 0.353     & 0.761 &           &       \\
\rowcolor{white} 7 & 0.996  &10.899 & 0.808     & 3.156 & 0.480     & 1.297 & 0.205     &       \\
\rowcolor{lgray} 8 & 0.997  &14.240 & 0.851     & 4.266 & 0.584     & 1.956 & 0.265     & 0.753 \\
\end{array} \]
\caption{Filterkoeffizienten für Tschebyscheff mit $1 dB$ Welligkeit}
\label{tab:filt-coeff-tsche1}
\end{table}

\clearpage
\subsection{Denormierung}
Tiefpass
\[ f_{0_n} = f_{0_{\text{Filter}}} \cdot f_{0_{n_{\text{Tabelle}}}} \]
Hochpass
\[ f_{0_n} = \frac{f_{0_{\text{Filter}}}}{f_{0_{n_{\text{Tabelle}}}}} \]