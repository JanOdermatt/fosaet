% coding:utf-8

%FOSAET, a LaTeX-Code for a electrical summary of basic electronics
%Copyright (C) 2013, Daniel Winz, Ervin Mazlagic

%This program is free software; you can redistribute it and/or
%modify it under the terms of the GNU General Public License
%as published by the Free Software Foundation; either version 2
%of the License, or (at your option) any later version.

%This program is distributed in the hope that it will be useful,
%but WITHOUT ANY WARRANTY; without even the implied warranty of
%MERCHANTABILITY or FITNESS FOR A PARTICULAR PURPOSE.  See the
%GNU General Public License for more details.
%----------------------------------------

\subsection{Frequenzabhängigkeit}
% \begin{figure}{h!}
%   \begin{tikzpicture}[circuit ee IEC]
%     \draw (0,0) to[V, l=V$_\text{q}$, ] (0,4);
%     \draw (2,4) to[R=$100\Omega$, -*] (2,2);
%     \draw (2,2) to[D, l=D, *-*] (2,0);
%     \draw (0,4) to[short, -*] (2,4);
%     \draw (0,0) to[short, -*] (2,0);
%     \draw (2,4) to[short, *-o] (4,4);
%     \draw (2,2) to[short, *-o] (4,2);
%     \draw (2,0) to[short, *-o] (4,0);
%     \draw (4,4) node[anchor=west] {CH1};
%     \draw (4,2) node[anchor=west] {CH2};
%     \draw (4,0) node[anchor=west] {GND};
%   \draw   (6,2) node[anchor=west] {KO};
%     \draw (2,0) node[ground] {};

%   \end{tikzpicture}
%   \caption{Ersatzschaltbild bei hohen Frequenzen}
% \end{figure}

\begin{figure}[h!]
  \centering
  \begin{circuitikz}[scale=1]\draw
    (0,0) to[short, o-*] (1,0)
    (1,0) to[short, *-] (1,1)
    (5,0) to[short, *-o] (6,0)
    (5,1) to[short, -*] (5,0)
    (1,0) to[R=R, *-] (3,0)
    (3,0) to[L=L, -*] (5,0)
    (1,1) to[C=C, ] (5,1)
    ;

    % (0,0) node[anchor=east] {GND}
    % (0,0) node[anchor=east] {GND}
    % (0,0) to[short, o-*] (2,0)
%     (0,0) to[V, l=V$_\text{q}$, ] (0,4)
%     (2,4) to[R=$100\Omega$, -*] (2,2)
%     (2,2) to[D, l=D, *-*] (2,0)
%     (0,4) to[short, -*] (2,4)
%     (0,0) to[short, -*] (2,0)
%     (2,4) to[short, *-o] (4,4)
%     (2,2) to[short, *-o] (4,2)
%     (2,0) to[short, *-o] (4,0)
%     (4,4) node[anchor=west] {CH1}
%     (4,2) node[anchor=west] {CH2}
%     (4,0) node[anchor=west] {GND}
%     (6,2) node[anchor=west] {KO}
%     (2,0) node[ground] {};
  \end{circuitikz}
  \caption{Ersatzschaltbild bei hohen Frequenzen}
\end{figure}
