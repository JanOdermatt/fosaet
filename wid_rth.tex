% coding:utf-8

%FOSAET, a LaTeX-Code for a electrical summary of basic electronics
%Copyright (C) 2013, Daniel Winz, Ervin Mazlagic

%This program is free software; you can redistribute it and/or
%modify it under the terms of the GNU General Public License
%as published by the Free Software Foundation; either version 2
%of the License, or (at your option) any later version.

%This program is distributed in the hope that it will be useful,
%but WITHOUT ANY WARRANTY; without even the implied warranty of
%MERCHANTABILITY or FITNESS FOR A PARTICULAR PURPOSE.  See the
%GNU General Public License for more details.
%----------------------------------------

\newpage
\section{Thermisches Ersatzschaltbild}
\begin{figure}[h!]
  \centering
  \begin{circuitikz}[scale=1]\draw
%     (0,0) to[short, o-] (0,0)
%     (0,0) to[short, -o] (0,0)
    (0,6) to[R=$R_{th_{JC}}$, o-o] (0,4)
    (0,4) to[R=$R_{th_{CH}}$, o-o] (0,2)
    (0,2) to[R=$R_{th_{HA}}$, o-o] (0,0)
    (0.4,6) node {J}
    (0.4,4) node {C}
    (0.4,2) node {H}
    (0.4,0) node {A}
    ;
  \end{circuitikz}
  \caption{Thermisches Ersatzschaltbild}
\end{figure}
Äquivalent zu Ohmschem Gesetz: 
\[ \Delta\vartheta = R_{th} \cdot P_v \]
\[ R_{th} = \frac{\ell}{\lambda \cdot A} \]
\begin{tabular}{@{}ll}
  $J$: & Junction $\rightarrow$ Sperrschicht \\
  $C$: & Case $\rightarrow$ Gehäuse \\
  $H$: & Heatsink $\rightarrow$ Kühlkörper \\
  $A$: & Ambient $\rightarrow$ Umgebung (Achtung! Meist innerhalb Gerät) \\
  $\lambda$: & Thermische Leitfähigkeit
\end{tabular}
