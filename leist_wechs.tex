% coding:utf-8

%FOSAET, a LaTeX-Code for a electrical summary of basic electronics
%Copyright (C) 2013, Daniel Winz, Ervin Mazlagic

%This program is free software; you can redistribute it and/or
%modify it under the terms of the GNU General Public License
%as published by the Free Software Foundation; either version 2
%of the License, or (at your option) any later version.

%This program is distributed in the hope that it will be useful,
%but WITHOUT ANY WARRANTY; without even the implied warranty of
%MERCHANTABILITY or FITNESS FOR A PARTICULAR PURPOSE.  See the
%GNU General Public License for more details.
%----------------------------------------

\section{Leistungsberechnung bei Wechselgrössen}

\subsection{Scheinleistung ($S$)}
\[ S = U_{eff} \cdot I_{eff} \]
\[ S = \sqrt{P^2 + Q^2} \qquad \cos(\varphi) = \frac{P}{S} \]
\[ \underline{S} = P + j Q \]
\[ \underline{S} = S_{eff} \cdot e^{j\varphi} \]
\[ \underline{S} = S_{eff} \angle \varphi \]
\[ \underline{S} = U \cdot I^* = ||I||^2 \cdot \underline{Z} 
= \frac{||U||^2}{\underline{Z}^*} \]

\subsection{Wirkleistung $P$}
\[ P = U_{eff} \cdot I_{eff} \cdot \cos(\varphi) \]

\subsection{Blindleistung $Q$}
\[ Q = U_{eff} \cdot I_{eff} \cdot sin(\varphi) \]

\section{Leistung bei nicht sinusförmigen Grössen}
Momentane Leistung
\[ p(t) = u \cdot i \]
Wirkleistung
\[ P = \frac{1}{T} \int_{0}^{T} p(t) dt 
= \frac{1}{T} \int_{0}^{T} u \cdot i ~ dt \]
Wirkleistung mit Fourierreihe
\[ P = \sum_{v=0}^{\infty} \frac{\hat{u}_v 
\cdot \hat{i}_v}{2} \cdot \cos(\varphi_v) \]
\[ \varphi_v = \varphi_{u_v} - \varphi_{i_v} 
\qquad \text{mit gleichen Frequenzen von $u$ und $i$} \]
Scheinleistung
\[ S = U \cdot I \]
Leistungsfaktor
\[ \lambda = \frac{P}{S} \]
Blindleistung
\[ Q^2 = S^2 - P^2 \]
Spezialfall: $u = \hat{U} \cdot \sin(\omega_1 \cdot t) ~ \text{und} ~ i 
= I_{AV} + \sum\limits_{v=1}^{\infty} \hat{I}_v \cdot 
\sin(v\cdot \omega_1 \cdot t + \varphi_v)$
\[ \begin{array}{@{}ll}
P = U \cdot I \cdot \cos(\varphi_1) \\
\varphi_1 = \varphi_u - \varphi_{i_1} \\
Q_1 = U \cdot I_1 \cdot \sin(\varphi_1) &\text{Grundschwingungsblindleistung} \\
S_1 = U \cdot I_1 &\text{Grundschwingungsscheinleistung} \\
S^2 = P^2 + {Q_1}^2 + D^2 \\
D = {Q_2} + {Q_3} + \dots + {Q_\infty} & \text{Verzerrungsblindleistung} \\
\end{array}  \]