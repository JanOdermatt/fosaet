% coding:utf-8

%FOSAET, a LaTeX-Code for a electrical summary of basic electronics
%Copyright (C) 2013, Daniel Winz, Ervin Mazlagic

%This program is free software; you can redistribute it and/or
%modify it under the terms of the GNU General Public License
%as published by the Free Software Foundation; either version 2
%of the License, or (at your option) any later version.

%This program is distributed in the hope that it will be useful,
%but WITHOUT ANY WARRANTY; without even the implied warranty of
%MERCHANTABILITY or FITNESS FOR A PARTICULAR PURPOSE.  See the
%GNU General Public License for more details.
%----------------------------------------

\subsection{Temperaturabhängigkeit von Widerständen}

\subsubsection{Lineare Temperaturabhängigkeit von Widerständen}
\[ R_\vartheta = R_{20} \cdot (1 + \alpha \cdot \Delta \vartheta) \]
\[ \Delta R = R_{20} \cdot \alpha \cdot \Delta \vartheta \]
Falls $R_{20}$ nicht bekannt ist, kann mit $R_A$ bei $\vartheta_a$ und $\tau$ 
die Temperaturabhängigkeit mit folgender Formel berechnet werden:  
\[ R_\vartheta = R_A \frac{\tau + \vartheta}{\tau + \vartheta_A} \]
Wobei 
\[ \tau = \frac{1}{\alpha_20} - 20^{\circ}\text{C} \]

\subsubsection{Nichtlineare Temperaturabhängigkeit (PTC)}
\[ R_N = 2 \cdot R_A \]
\[ R_T = R_N \cdot e^{\alpha (T - T_N)} \]
\[ \alpha = \frac{\ln R_1 - \ln R_2}{T_2 - T_1} 
= \frac{\ln\left(\frac{R_1}{R_2}\right)}{\Delta T} 
= \frac{d R_t}{d T}\frac{1}{R_T} \]
\begin{tabular}{@{}lp{0.5\textwidth}}
  $t_A$:        & Anfangstemperatur $[^\circ C]$ \\
  $t_N$:        & Nenntemperatur $[^\circ C]$ \\
  $R_N$:        & Anfangswiderstand $[\Omega]$ \\
  $\alpha$:     & Temperaturkoeffizient oberhalb der Nenntemperatur 
                  ($+5\frac{\%}{K}$ bis $+70\frac{\%}{K}$) \\
  $R_1$, $R_2$: & Widerstände $[\Omega]$ für zwei Punkte oberhalb der 
                  Nenntemperatur \\
  $\Delta T$:   & Temperaturunterschied $[^\circ C]$ für die beiden Punkte mit 
                  $R_1$ und $R_2$
\end{tabular}

\subsubsection{Nichtlineare Temperaturabhängigkeit (NTC)}
\[ R_T = R_N \cdot e^{b\left(\frac{1}{T} - \frac{1}{T_N}\right)} 
= A \cdot e^{\frac{B}{T}} \]
\[ \alpha_T = \frac{d R_T}{d T}\frac{1}{R_T} = -\frac{b}{T^2} \]
\begin{tabular}{@{}lp{0.5\textwidth}}
  $R_T$:        & Heissleiterwiderstand bei $T[K]$ in $[\Omega]$ \\
  $R_N$:        & R bei Bezugstemperatur 
                  ($293 K$ entsprechen $20^\circ C$ $[\Omega]$) \\
  $T$:          & Betriebs-, Umgebungstemperatur $[K]$ \\
  $T_N$:        & Bezugstemperatur $[K]$ \\
  $A$:          & Formkonstante $[\Omega]$ nach Herstellerangaben \\
  $B$:          & Regelkonstante $[K]$ \\
  $\alpha_T$:   & temperaturabhängiger Temperaturkoeffizient bei $T$ $[K]$
\end{tabular}